\title{Reading List}
\author{
        Mihaela Apetroaie-Cristea \\
        University of Southampton\\
         \underline{Southampton}
}
\date{\today}

\documentclass[12pt]{article}

\begin{document}
\maketitle



\section{Internet of Things International Conferences}


\subsection {IoT 2015 Conference} -
The 5th International Conference on the Internet of Things (IoT 2015) 


\paragraph{When and Where} October 26 - 28, 2015 in Seoul, S. Korea.

 \paragraph{Theme}The Internet of Things Conference is seeking original, high impact research papers on all topics related to the development of the Internet of Things. Papers will be reviewed and selected based on technical novelty, integrity of the analysis and social impacts and practical relevance.

\paragraph{Submission Deadline}
\begin{enumerate}
\item\ Abstract Registration: May 22, 2015
\item\ Paper Submission: June 4, 2015
\item\ Acceptance notice: July 31, 2015

\end{enumerate}

\subsection {HealthyIoT 2015} - IoT360
\paragraph {When and where}
\paragraph {Submission deadline} 30th June 2015
\paragraph {Theme} Internet of Things and Healthcare - electronic healthcare, patient care and medical data management.

\subsection {CoIoT - IoT360} - 2nd International Conference on Cognitive Internet of Things Technologies
\paragraph {When and where}  October 26 – 27, 2015 Rome, Italy
\paragraph {Theme} Cognitive IoT aims at gathering enthusiastic researchers and practitioners from AI and IoT-related areas sharing the common goal of addressing the new challenges posed by the Cognitive aspect of IoT, by using new or leveraging existing Artificial Intelligence techniques.

\paragraph {Submission deadline}
\begin{enumerate}
\item\ Full Paper Submission deadline: July 1, 2015
\item\ PCamera-ready deadline: August 25, 2015
\item\ Start of Conference: October 26, 2015
\item\ End of Conference: October 25, 2015
\end{enumerate}

\subsection{FiCloud 2015 } - The 3rd International Conference on Future Internet of Things and Cloud
\paragraph {When and where} Aug 24, 2015 - Aug 26, 2015, Rome, Italy
\paragraph {Theme} The theme of this conference is to promote the state of the art in scientific and practical research of the IoT and cloud computing. It provides a forum for bringing together researchers and practitioners from academia, industry, and public sector in an effort to present their research work and share research and development ideas in the area of IoT and cloud computing. 
\paragraph {Submission Deadlines} 
\begin{enumerate}
\item\ Submission Deadline:	Apr 16, 2015
\item\ Notification Due	May: 30, 2015
\item\ Final Version Due: Jun 20, 2015
\end{enumerate}
\subsection{IEEE on Internet of Things} 
\paragraph {When and where} 14 - 16th December 2015 Milan, Italy
\paragraph {Theme} The 2015 IEEE 2nd World Forum on Internet of Things (WF-IoT) - Technologies, Applications and Social Implications is a unique event for industry leaders, academics and decision making government officials. This event is designed to examine key critical innovations across technologies which will alter the research and application space of the future. The Internet of Things envisions a highly networked future, where every object is integrated to interact with each other, allowing for communications between objects, as well as between humans and objects, which enables the control of intelligent systems in our daily lives.
\paragraph {Deadlines}
\begin{enumerate}
\item\ Submission deadline: 8th July 2015
\item\ Acceptance notification: 15th September 2015

\end{enumerate}


\section{Internet of Things Interesting Projects}

\subsection{An application for Health System that will allow everyone see relevant information about the patients depending on whom is accessing it} 
\paragraph {} Powered by District Defend, the tablet could be set to access the patient's records only when the device detects the presence—through RFID-based identity tags, for instance—of both the doctor and the patient.
      "Right now [ hospitals], it's all or nothing, in most cases," Smith states. "If I am a doctor, I should have access to a patient's health records. If I'm the guy who needs to clean the bedpan, all I need to know is when I need to go in and clean a room." Using District Defend, mobile devices could deliver data based on who is using the tablet and where he or she is located.
 \subsection{Inteligitics} uses temperature, humidity and light sensors that transmit data via ZigBee and Bluetooth transceivers to help logistics companies monitor the condition of fruits and vegetables in transit, in order to keep them in optimal conditions and extend shelf life. [http://www.iotjournal.com/articles/view?13044]

  \subsection {Olea Sensor Networks}, a Sunnyvale, Calif.-based startup, introduced its sensor platform, made to enable end users to differentiate animate from inanimate objects—and even to determine whether animate objects are human or non-human. The company is now working with a number of businesses on pilot programs to test safety applications. [http://www.iotjournal.com/articles/view?13016]. NASA recently deployed two prototype devices that utilize technology similar to Olea Sensors Networks' in Nepal, to search under rubble for survivors of the 7.8 magnitude earthquake. The NASA technology, known as Finding Individuals for Disaster and Emergency Response (FINDER), led rescuers to four survivors—and this was the first time it managed to save lives in a real emergency.



http://www.iotjournal.com/articles/view?12992 help growing in-doors low potassium lettuce.




 The Tennessee Aquarium, in Chattanooga, is employing Bluetooth Low Energy (BLE) beacon technology to allow young visitors to play the role of scientist, by sending them alerts via the aquarium app on their smartphones. When a phone comes within range of the beacon mounted at nearly two dozen of its exhibits, the app receives a card asking the child to take note of a certain animal and its behaviors. The system was taken live two weeks ago. [http://www.iotjournal.com/articles/view?12851]. The aquarium app, available for download at both Google Play andiTunes, provides visitors with information before, during and after their visit, such as where to park, hours of service, scheduled daily events and details regarding the animals living at the facility.



Sephora to Go mobile app (available at the iTunes and Google Play websites) is designed to notify app users of any new promotions and birthday month benefits, and to remind them of in-store services such as makeovers. The iOS version of the app also enables users to receive special offers and information triggered by beacons installed within its stores. [http://www.iotjournal.com/articles/view?12819]
 \section {List with harware}

\bibliographystyle{abbrv}
\bibliography{main}

\end{document}
This is never printed
  